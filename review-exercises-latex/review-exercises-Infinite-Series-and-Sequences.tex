\documentclass[letterstyle,12pt]{extarticle}
\usepackage[utf8]{inputenc}
\usepackage[nomarginpar, margin=0.5in]{geometry}
\usepackage{tabularx}
\usepackage[document]{ragged2e}
\usepackage{setspace}
\usepackage{amsmath}
\usepackage{amssymb}
\usepackage{xcolor}
\usepackage{mlmodern}
\usepackage{float}
\usepackage{supertabular}
% \usepackage[T1]{fontenc}
\usepackage{mathptmx}
\linespread{1.3} % Set line spacing 
\usepackage{pgfplots}
\usepackage{bigstrut}
\usepackage[ngerman]{babel}
\usepackage{mathtools}
\usepackage[T1]{fontenc}
\usepackage{enumitem}
\newcounter{qcounter}
\usepackage{multicol}
\pgfplotsset{compat = newest}
\usepackage{fancyhdr}
\everymath{\displaystyle}
\definecolor{agua}{RGB}{90,160,254}
\newcommand{\nspace}{\hspace{2.52mm}}

\newcommand\Tstrut{\smash[b]{\strut}}
\newcommand\Bstrut{\smash[t]{\strut}}
\newcommand{\dd}{\text{d}}
\newcommand{\di}{\, \text{d}}

\newcommand{\oldchoices}[1]{
\vspace{0.8em} 
\begin{enumerate}[label=(\Alph*)]
\setlength\itemsep{1em} #1 \interlinepenalty10000 
\end{enumerate}
}

\newcommand{\choices}[5]{
\vspace{0.8em} 
\begin{enumerate}[label=(\Alph*)]
\setlength\itemsep{1em} 
\item
#1 
\item 
#2
\item
#3
\item
#4
\item
#5
\end{enumerate}
}


\newcommand{\choicesline}[5]{    
\vspace{2em} \break 
\begin{tabularx}{0.95 \textwidth} { 
>{\arraybackslash}X 
>{\arraybackslash}X 
>{\arraybackslash}X 
>{\arraybackslash}X 
>{\arraybackslash}X }
(A) \; #1
& 
(B) \; #2
& 
(C) \; #3
& 
(D) \; #4
&
(E) \; #5 
\end{tabularx}
\vspace{2em} \break
}

\pgfplotsset{every axis/.append style={
    axis on top = true,
    width = \textwidth,  
    height =  \textwidth,
    clip=false, 
    axis line style = ultra thick, 
    axis lines = middle,
    label style={font=\Huge}, %x- and y-label sizes
    x label style={at={(axis description cs:1.04,0.03)},anchor=north},
    y label style={at={(axis description cs:-0.135,1.13)}}
    every tick label/.append style={font=\Huge},
    every major tick/.append style={very thick, major tick length=10pt},
    tick style = {xticklabel style={yshift=-0.2em, anchor=south}}
    }}
\pgfplotsset{tick style = {xticklabel style={yshift=-4ex, anchor=south}}}
\newcommand{\drawOrigin}{\draw (0, 0) node[left, below, xshift = -0.9em] {\Large $O$}} % draws the "O" marking the origin

\newcommand{\plotCircle}[4]{\draw #1 circle (#4 pt) [#2, fill = #3]}

\newcommand{\qspace}[1]{\begin{spacing}{1.8}#1\end{spacing}}
\newcommand{\ans}[1]{{\color{black} #1}}

\newenvironment{question}
    {\begin{minipage}{0.9 \textwidth}
        \item
    }
    { 
    \end{minipage} \vspace{4ex}
    }
    %

\newcommand{\bq}{\begin{question}}
\newcommand{\eq}{\end{question}}

\newcommand{\ques}[1]{\begin{minipage}{0.9 \textwidth} \item #1 \end{minipage} \vspace{4ex}}
\newcommand{\romanlist}[1]{\begin{enumerate}[label=\Roman*., leftmargin=15mm] #1 \end{enumerate}}

   
\pagestyle{fancy}
\lhead{\begin{figure}[H] \vspace{10ex} \includegraphics[scale=0.1]{logo-light.png} \vspace{-7.3ex} \end{figure}}
\rhead{\large \textsc{Review Exercises—Infinite Series and Sequences}}
\fancyfoot{}
\lfoot{\copyright \textsc{Valculus} 2023}
\cfoot{\textbf{-\thepage-}}

\geometry{
    head = 1.1in,
    bottom = 1.3in,
    left = 0.5in,
    right = 0.5in
    }
    
\renewcommand{\headrulewidth}{0.3pt}
\renewcommand{\footrulewidth}{0.4pt}
\setlength{\headsep}{0.9in}

\begin{document}
\begin{center}
    {\bf
    \MakeUppercase{Infinite Series and Sequences\\[1ex]
    NUMBER OF QUESTIONS—45 \\[1ex] 
    NO CALCULATOR \\[5ex]
    }}
\end{center}

\begin{list}{\textbf{\arabic{qcounter}.}~}{\usecounter{qcounter}}
\setlength\itemsep{3em}

\begin{question}
What is the difference between a sequence and a series? 
\choices
{A sequence is the sum of a set of numbers, whereas a series is a set of numbers.}
{A sequence is a set of numbers arranged in ascending order, whereas a series is the sum of a set of numbers arranged in ascending order.}
{\ans{A sequence is a set of numbers, whereas a series is the sum of a set of numbers.} }
{A sequence always diverges, whereas a series may converge or diverge.}
{A sequence always converges, whereas a series may converge or diverge.}
\end{question}

\begin{question}
Which sequence converges?
\choices
{\(\left\{-1, 1, -1, 1, -1, \dots\right\}\)}
{\(\left\{-2, -1, 0, 1, 2, \dots\right\}\)}
{\(\left\{1, 2, 4, 8, 16, \dots\right\}\)}
{\ans{\(\left\{1, \frac{3}{4}, \frac{9}{16}, \frac{27}{64}, \frac{81}{256}, \dots \right\}\)}}
{\(\left\{\frac{1}{2}, -\frac{3}{2}, \frac{5}{2}, -\frac{7}{2}, \frac{9}{2}, \dots\right\}\)}
\end{question}



\begin{question}
\(\frac{\pi}{4} - \frac{\pi^2}{16} + \frac{\pi^3}{64} - \frac{\pi^4}{256} + \cdots + (-1)^{n + 1} \left(\frac{\pi}{4}\right)^n + \cdots\) \; is \\
\choicesline
{\(\frac{-4}{4 + \pi}\)}
{\(\frac{\pi}{4 - \pi}\)}
{\ans{\(\frac{\pi}{4 + \pi}\)}}
{\(\frac{4}{4 - \pi}\)}
{divergent}
\end{question}

\begin{question}
By the Binomial Theorem, \((3x - 5)^4\) is 
\choices
{\ans{\((3x)^4 + {4 \choose 1} (3x)^3 (-5) + {4 \choose 2} (3x)^2 (-5)^2 + {4 \choose 3} (3x) (-5)^3 + (-5)^4\)}}
{\(-(3x)^4 - {4 \choose 1} (3x)^3 (-5) - {4 \choose 2} (3x)^2 (-5)^2 - {4 \choose 3} (3x) (-5)^3 - (-5)^4\)}
{\((3x)^4 + {4 \choose 1} (3x)^3 (5) + {4 \choose 2} (3x)^2 (5)^2 + {4 \choose 3} (3x) (5)^3 + (5)^4\)}
{\((3x)^4 + {4 \choose 1} - (3x)^3 (5) - {4 \choose 2} (3x)^2 (5)^2 - {4 \choose 3} (3x) (5)^3 - (5)^4\)}
{\((3x)^4 + {4 \choose 1} - (3x)^3 (-5) - {4 \choose 2} (3x)^2 (-5)^2 - {4 \choose 3} (3x) (-5)^3 - (-5)^4\)}
\end{question}

\begin{question}
The Maclaurin series of \(\cos x\) is 
\choices
{\(1 + \frac{x^2}{2} + \frac{x^4}{4!} + \frac{x^6}{6!} + \cdots + \frac{x^{2n}}{(2n)!} + \cdots\)}
{\(x - \frac{x^3}{3!} + \frac{x^5}{5!} - \frac{x^7}{7!} + \cdots + \frac{(-1)^n x^{2n + 1}}{(2n + 1)!} + \cdots\)}
{\(x + \frac{x^3}{3!} + \frac{x^5}{5!} + \frac{x^7}{7!} + \cdots + \frac{x^{2n + 1}}{(2n + 1)!} + \cdots\)}
{\(1 + x + \frac{x^2}{2} + \frac{x^3}{3!} + \cdots + \frac{x^n}{n!} + \cdots\)}
{\ans{\(1 - \frac{x^2}{2} + \frac{x^4}{4!} - \frac{x^6}{6!} + \cdots + \frac{(-1)^n \, x^{2n}}{(2n)!} + \cdots\)}}
\end{question}


\begin{question}
Let \(a_n = \frac{1}{\sqrt[3]{n^2}}.\) Which option is true? 
\choices
{\(\sum_{n = 1}^\infty a_n\) converges because \(\lim_{n \to \infty} a_n = 0.\)}
{\(\sum_{n = 1}^\infty a_n\) converges because it is a \(p\)-series with \(p \leqslant 1.\)}
{\(\sum_{n = 1}^\infty a_n\) diverges because \(\lim_{n \to \infty} a_n = 0.\)}
{\ans{\(\sum_{n = 1}^\infty a_n\) diverges because it is a \(p\)-series with \(p \leqslant 1.\)}}
{It cannot be determined from the given information.}
\end{question}

\begin{question}
Which series diverges?
\choicesline
{\(\sum_{n = 1}^\infty \frac{1}{\sqrt{n^5}}\)}
{\(\sum_{n = 1}^\infty \frac{1}{n^2}\)}
{\ans{\(\sum_{n = 1}^\infty \frac{1}{\sqrt[3]{n}}\)}}
{\(\sum_{n = 1}^\infty \frac{1}{n^{5/4}}\)}
{\(\sum_{n = 1}^\infty \frac{n}{n^2 \sqrt n}\)}
\end{question}

\begin{question}
Suppose that \(a_n \geqslant b_n > 0\) and \(\{b_n\}\) diverges. Which of the following must be true?

\romanlist{
\item
\(\{a_n\}\) diverges.
\item 
\(\lim_{n \to \infty} a_n = \infty.\)
\item
\(\lim_{n \to \infty} b_n = \infty.\)
}

\choices
{\ans{None}}
{I only}
{I and II only}
{I and III only}
{I, II, and III}
\end{question}

\begin{question}
Which series is alternating?
\choices
{\(\sum_{n = 1}^\infty (-1)^{2n}\)}
{\(\sum_{n = 1}^\infty (-1)^n \cos(\pi n)\)}
{\(\sum_{n = 1}^\infty \frac{(-1)^{2n + 3}}{n}\)}
{\ans{\(\sum_{n = 1}^\infty \frac{\sin(\pi n)}{\sqrt n}\)}}
{All of the above}
\end{question}

\begin{question}
Suppose \(\sum_{n = 1}^\infty a_n\) converges. Which of the following must be true?

\romanlist{
\item
\(\lim_{n \to \infty} a_n = 0\)
\item
The sequence \(\{a_n\}\) converges.
\item
\(\lim_{N \to \infty} \sum_{n = 1}^N a_n = \pm \infty.\)
}

\choices
{I only}
{II only}
{\ans{I and II only}}
{II and III only}
{I, II, and III}
\end{question}

\begin{question}
The Maclaurin series \(x^2 -\frac{x^4}{3!} + \frac{x^6}{5!} - \frac{x^8}{7!} + \cdots\) 
converges to which function?
\choicesline
{\(\sin x\)}
{\(\cos x\)}
{\(x^2 e^x\)}
{\(x^2 \sin x\)}
{\ans{\(x \sin x\)}}
\end{question}

\begin{question}
Which series converges? 
\choicesline
{\(\sum_{n = 1}^\infty (-1)^n\)}
{\ans{\(\sum_{n = 1}^\infty \frac{(-1)^n}{\sqrt n}\)}}
{\(\sum_{n = 1}^\infty (-1)^n \, n\)}
{\(\sum_{n = 1}^\infty (-1)^{2n} \)}
{\(\sum_{n = 1}^\infty \frac{(-1)^{2n}}{n}\)}
\end{question}

\begin{question}
Suppose that \(\sum_{n = 1}^\infty b_n\) converges and \(a_n \leqslant b_n\) for \(n \geqslant 1.\) Which choice must be true?
\choices
{\(\sum_{n = 1}^\infty a_n\) converges.}
{\(\sum_{n = 1}^\infty a_n\) diverges.}
{\(\lim_{n \to \infty} a_n = 0.\)}
{\(\lim_{n \to \infty} b_n \ne 0.\)}
{\ans{None of the above}}
\end{question}


\begin{question}
Function \(g\) has derivatives of all orders. The table below shows selected values of the derivatives of \(g(x)\) at \(x = 3,\) with \(g(3) = 1.\) What is the fourth-degree Taylor series of \(g(x)\) centered at \(x = 3 \; ?\) \\[1.6em]
\renewcommand{\arraystretch}{1.1}
\begin{center}
\begin{tabularx}{0.7\textwidth} { 
    | >{\centering\arraybackslash}X 
    || >{\centering\arraybackslash}X 
    | >{\centering\arraybackslash}X 
    | >{\centering\arraybackslash}X 
    | >{\centering\arraybackslash}X 
    | >{\centering\arraybackslash}X | }
    \hline
    \(n\) & 1 & 2 & 3 & 4  \\
    \hline
    \(g^{(n)}(3)\) & \(-2\) & \(2\) & \(3/2\) & \(-4/3\) \\
    \hline
\end{tabularx}
\end{center}

\choices
{\ans{\(1 -2(x - 3) + (x - 3)^2 + \frac{1}{4} (x - 3)^3 - \frac{1}{18} (x - 3)^4\)}}
{\(1 -2(x - 3) + 2(x - 3)^2 + \frac{3}{2}(x - 3)^3 - \frac{4}{3} (x - 3)^4\)}
{\(-2(x - 3) + 2(x - 3)^2 + \frac{3}{2}(x - 3)^3 - \frac{4}{3} (x - 3)^4\)}
{\(-2(x - 3) + (x - 3)^2 + \frac{1}{4} (x - 3)^3 - \frac{1}{18} (x - 3)^4\)}
{\(1 -2(x + 3) + (x + 3)^2 + \frac{1}{4} (x + 3)^3 - \frac{1}{18} (x + 3)^4\)}
\end{question}

\begin{question}
For what values of \(k\) does \(\sum_{n = 1}^\infty \frac{1}{n^{2k - 3}}\) diverge?
\choices
{\ans{\(k \leqslant 2\)}}
{\(k < \frac{3}{2}\)}
{\(k > \frac{3}{2}\)}
{\(k > 2\)}
{\(k \geqslant 2\)}
\end{question}

\begin{question}
Which choice correctly describes the convergence or divergence of \(S = \sum_{n = 1}^\infty \frac{1}{\sqrt{n^2 + 3}} \; ?\)
\choices
{\(S\) converges by comparison with \(\sum_{n = 1}^\infty \frac{1}{n^2}.\)}
{\(S\) diverges by comparison with \(\sum_{n = 1}^\infty \frac{1}{n^2}.\)}
{\(S\) converges by comparison with \(\sum_{n = 1}^\infty \frac{1}{n}.\)}
{\ans{\(S\) diverges by comparison with \(\sum_{n = 1}^\infty \frac{1}{n}.\)}}
{It cannot be determined from the given information.}
\end{question}

\begin{question}
For all \(x,\) \(1 + \frac{x^2}{2!} + \frac{x^4}{4!} + \cdots + \frac{x^{2n}}{(2n)!} + \cdots\) \; converges to \\
\choicesline
{\(\sin x\)}
{\ans{\(\cosh x\)}}
{\(\sinh x\)}
{\(\cos x\)}
{\(e^{2x}\)}
\end{question}

\begin{question}
\qspace{
Let \(f\) be a function with derivatives of all orders. The fourth-degree Taylor series of \(f(x)\) centered at \(x = 2\) is \(2 + (x - 2) - \frac{2}{5}(x - 2)^2 - \frac{3}{2} (x - 2)^3 + \frac{1}{7} (x - 2)^4 + \cdots.\) What is the value of \(f'''(2) \; ?\)}
\choicesline
{\ans{\(-9\)}}
{\(-\frac{9}{2}\)}
{\(-\frac{3}{2}\)}
{\(-\frac{1}{4}\)}
{0}
\end{question}

\begin{question}
The coefficient of \(x^3\) in the Maclaurin series of \(e^{-x/3}\) is 
\choicesline
{\(-\frac{1}{27}\)}
{\ans{\(- \frac{1}{162}\)}}
{\(\nspace \frac{1}{162}\)}
{\( \nspace \frac{1}{27}\)}
{\( \nspace \frac{1}{6}\)}
\end{question}

\begin{question}
Which series converges conditionally?
\choices
{\(\sum_{n = 1}^\infty \frac{1}{n + 1}\)}
{\(\sum_{n = 1}^\infty \frac{(-1)^n}{n^2}\)}
{\ans{\(\sum_{n = 1}^\infty \frac{(-1)^n}{\sqrt[3]{n + 1}}\)}}
{\(\sum_{n = 1}^\infty \frac{1}{\sqrt{n}}\)}
{\(\sum_{n = 1}^\infty \frac{(-1)^n}{\sqrt{n^3 + 2}}\)}
\end{question}


\begin{question}
Which choice about \(\sum_{n = 1}^\infty \frac{1}{\sqrt{2n + 3}}\) is correct?
\choices
{The series converges because \(\int_1^\infty \frac{1}{\sqrt{2x + 3}} \di x\) converges.}
{The series converges because \(\int_1^\infty \frac{1}{\sqrt{2x + 3}} \di x\) diverges.} 
{The series converges because \(\int_1^\infty \frac{1}{\sqrt{2x + 3}} \di x > 0.\)}
{The series diverges because \(\int_1^\infty \frac{1}{\sqrt{2x + 3}} \di x\) converges.} 
{\ans{The series diverges because \(\int_1^\infty \frac{1}{\sqrt{2x + 3}} \di x\) diverges.}}
\end{question}

\begin{question}
The binomial series expansion of \(\frac{1}{2} \sqrt{4 - x}\) is 
\choices
{\ans{\(1 + \frac{1}{2} \left(-\frac{x}{4}\right) + \frac{\frac{1}{2} \left(-\frac{1}{2}\right)}{2!} \left(-\frac{x}{4}\right)^2 + \frac{\frac{1}{2} \left(-\frac{1}{2}\right) \left(-\frac{3}{2}\right)}{3!} \left(-\frac{x}{4}\right)^3 + \cdots\)}}
{\(1 - \frac{1}{2} \left(-\frac{x}{4}\right) - \frac{\frac{1}{2} \left(-\frac{1}{2}\right)}{2!} \left(-\frac{x}{4}\right)^2 - \frac{\frac{1}{2} \left(-\frac{1}{2}\right) \left(-\frac{3}{2}\right)}{3!} \left(-\frac{x}{4}\right)^3 + \cdots\)}
{\(1 + \frac{1}{2} \left(\frac{x}{4}\right) + \frac{\frac{1}{2} \left(-\frac{1}{2}\right)}{2!} \left(\frac{x}{4}\right)^2 + \frac{\frac{1}{2} \left(-\frac{1}{2}\right) \left(-\frac{3}{2}\right)}{3!} \left(\frac{x}{4}\right)^3 + \cdots\)}
{\(1 + \frac{1}{2} \left(-\frac{x}{4}\right) + \frac{\frac{1}{2} \left(\frac{1}{2}\right)}{2!} \left(-\frac{x}{4}\right)^2 + \frac{\frac{1}{2} \left(\frac{1}{2}\right) \left(\frac{3}{2}\right)}{3!} \left(-\frac{x}{4}\right)^3 + \cdots\)}
{\(1 + \frac{1}{2} \left(\frac{x}{4}\right) + \frac{\frac{1}{2} \left(\frac{1}{2}\right)}{2!} \left(\frac{x}{4}\right)^2 + \frac{\frac{1}{2} \left(\frac{1}{2}\right) \left(\frac{3}{2}\right)}{3!} \left(\frac{x}{4}\right)^3 + \cdots\)}
\end{question}

\begin{question}
For which series is the Root Test inconclusive?
\choices
{\(\sum_{n = 1}^\infty \frac{5^n}{8^n + 1}\)}
{\ans{\(\sum_{n = 1}^\infty \left(\frac{n + 3}{n + 2}\right)^n\)}}
{\(\sum_{n = 1}^\infty \left(\frac{2n + 7}{3n - 1}\right)^n\)}
{\(\sum_{n = 1}^\infty \left(\frac{6n^2 + n - 2}{4n^2 + 9}\right)^n\)}
{\(\sum_{n = 1}^\infty \frac{n^n}{5^{3n + 1}}\)}
\end{question}

\begin{question}
Suppose that \(\sum_{n = 1}^\infty a_n\) converges and \(c\) is a finite constant. All the following series must converge \emph{except}
\choices
{\(c + \sum_{n = 1}^\infty a_n\)}
{\(c \sum_{n = 1}^\infty a_n\)}
{\(\sum_{n = 1 + |c|}^\infty a_n\)}
{\ans{\(\sum_{n = 1}^\infty [a_n]^c\)}}
{\(\sum_{n = 1}^\infty [a_n + c \, a_n]\)}
\end{question}

\begin{question}
The fourth-degree Taylor series of \(e^x\) centered at \(x = 2\) is 
\choices
{\(1 + (x + 2) + \frac{1}{2}(x + 2)^2 + \frac{1}{3!}(x + 2)^3 + \frac{1}{4!}(x + 2)^3\)}
{\(1 + (x - 2) + \frac{1}{2}(x - 2)^2 + \frac{1}{3!}(x - 2)^3 + \frac{1}{4!}(x - 2)^3\)}
{\(e^2 \left[1 + (x + 2) + \frac{1}{2}(x + 2)^2 + \frac{1}{3!}(x + 2)^3 + \frac{1}{4!}(x + 2)^3 \right]\)}
{\ans{\(e^2 \left[1 + (x - 2) + \frac{1}{2}(x - 2)^2 + \frac{1}{3!}(x - 2)^3 + \frac{1}{4!}(x - 2)^3 \right]\)}}
{\(e^2 \left[1 + (x - 2) + \frac{1}{2}(x - 2)^2 + \frac{1}{3}(x - 2)^3 + \frac{1}{4}(x - 2)^3 \right]\)}
\end{question}

\begin{question}
Which function does not have a Taylor series at the given center?
\choices
{\(e^x\) centered at \(x = e\)}
{\(\tan x\) centered at \(x = \frac{\pi}{3}\)}
{\(\sqrt x\) centered at \(x = 1\)}
{\(\ln x\) centered at \(x = 1\)}
{\ans{\(\sqrt{x - 1}\) centered at \(x = 0\)}}
\end{question}

\begin{question}
The radius of convergence of \(\sum_{n = 1}^\infty \frac{x^n \, n^2}{n!} \,\) is
\choicesline
{0}
{1}
{2}
{\(e\)}
{\ans{\(\infty\)}}
\end{question}

\begin{question}
Suppose that \(\sum_{n = 1}^\infty a_n \geqslant \sum_{n = 1}^\infty b_n > 0\) and \(\sum_{n = 1}^\infty b_n\) diverges. Which of the following must be true?

\romanlist{
\item
\(\sum_{n = 1}^\infty a_n\) diverges.
\item 
\(\sum_{n = 1}^\infty a_n\) must diverge to \(+\infty.\)
\item
\(\sum_{n = 1}^\infty b_n\) must diverge to \(+\infty.\)
}

\choices
{I only}
{I and II only }
{II and III only}
{I and III only}
{\ans{I, II, and III}}
\end{question}


\begin{question}
For \(-1 < x < 1,\) which series is equivalent to \(\frac{1}{1 - x^2} \; ?\)
\choices
{\(x^2 - x^4 + x^6 - x^8 + \cdots\)}
{\(x^2 + x^4 + x^6 + x^8 + \cdots\)}
{\ans{\(1 + x^2 + x^4 + x^6 + x^8 + \cdots\)}}
{\(1 - x^2 + x^4 + -x^6 + x^8 + \cdots\)}
{\(-1 - x^2 - x^4 - x^6 - x^8 - \cdots\)}
\end{question}

\begin{question}
Which series diverges?
\choices
{\(\sum_{n = 1}^\infty \frac{1}{\sqrt{2n^3 + 9}}\)}
{\(\sum_{n = 1}^\infty \frac{\sqrt{n^2 + 5}}{\sqrt{n^6 + 9}}\)}
{\ans{\(\sum_{n = 1}^\infty \frac{n + 2}{\sqrt{3n^4 + 4}}\)}}
{\(\sum_{n = 1}^\infty \frac{n}{\sqrt{n^6 + 2}}\)}
{\(\sum_{n = 1}^\infty \frac{\sqrt{2n + 1}}{\sqrt{5n^4 + 3}}\)}
\end{question}

\begin{question}
Let \(a_n = \frac{n^2 \sqrt{n^2 + 4}}{n^3}.\) Which choice is correct?
\choices
{Because \(\lim_{n \to \infty} \left|\frac{a_{n + 1}}{a_n}\right| > 1,\) \(\sum_{n = 1}^\infty a_n\) diverges by the Ratio Test.}
{Because \(\lim_{n \to \infty} \left|\frac{a_{n + 1}}{a_n}\right| < 1,\) \(\sum_{n = 1}^\infty a_n\) diverges by the Ratio Test.}
{Because \(\lim_{n \to \infty} \left|\frac{a_{n + 1}}{a_n}\right| < 1,\) \(\sum_{n = 1}^\infty a_n\) converges by the Ratio Test.}
{Because \(\lim_{n \to \infty} \left|\frac{a_{n + 1}}{a_n}\right| = 1,\) \(\sum_{n = 1}^\infty a_n\) diverges by the Ratio Test.}
{\ans{Because \(\lim_{n \to \infty} \left|\frac{a_{n + 1}}{a_n}\right| = 1,\) the Ratio Test for \(\sum_{n = 1}^\infty a_n\) is inconclusive.}}
\end{question}

\begin{question}
Function \(f\) has derivatives of all orders. At \(x = 0,\) \(f(x)\) is decreasing and concave up. Which choice could be the third-degree Maclaurin series for \(f \; ?\)
\choices
{\(2 + x - \frac{1}{3} x^2 + 4x^3\)}
{\(2 + x + \frac{1}{3} x^2 - 4x^3\)}
{\(-2 - x - \frac{1}{3} x^2 - 4x^3\)}
{\ans{\(-2 - x + \frac{1}{3} x^2 - 4x^3\)}}
{\(2 + x + \frac{1}{3} x^2 + 4x^3\)}
\end{question}

\begin{question}
Let \(a_n = \frac{1}{\sqrt{n + 2}}\) for \(n \geq 0.\) Let \(f\) be a continuous function such that \(f(n) = a_n.\) Which choice must be true?
\choices
{\ans{\(\sum_{n = 2}^{\infty} a_n \leqslant \int_1^\infty f(x) \di x\)}}
{\(\sum_{n = 2}^{\infty} a_n \geqslant \int_1^\infty f(x) \di x\)}
{\(\sum_{n = 2}^{\infty} a_n = \int_1^\infty f(x) \di x\)}
{\(\sum_{n = 1}^{\infty} a_n = \int_1^\infty f(x) \di x\)}
{It cannot be determined from the given information.}
\end{question}

\begin{question}
\(\frac{\text{d}}{\text{d} x} \sum_{n = 1}^\infty \frac{x^{3n + 1}}{n!} = \)
\choices
{\(\frac{x^5}{5} + \frac{x^8}{8(2!)} + \frac{x^{11}}{11(3!)} + \cdots + \frac{x^{3n + 2}}{(3n + 2) n!} + \cdots\)}
{\(4x + \frac{7}{2!} x^4 + \frac{10}{3!} x^7 + \cdots + (3n + 1) \frac{3^{3n - 2}}{n!} + \cdots\)}
{\(\frac{x^7}{7} + \frac{x^{10}}{10(2!)} + \frac{x^{13}}{13(3!)} + \cdots + \frac{x^{3n + 4}}{(3n + 4) n!} + \cdots \)}
{\(4x^3 + \frac{1}{2!} x^6 + \frac{1}{3!} x^9 + \cdots + \frac{x^{3n}}{n!} + \cdots\)}
{\ans{\(4x^3 + \frac{7}{2!} x^6 + \frac{10}{3!} x^9 + \cdots + (3n + 1) \frac{x^{3n}}{n!} + \cdots\)}}
\end{question}

\begin{question}
\(\sum_{n = 1}^\infty \left(\frac{1}{n^2 + n}\right)\) is
\choicesline
{0}
{\ans 1}
{\(\frac{3}{2}\)}
{2}
{divergent}
\end{question}

\begin{question}
Let \(S = \sum_{n = 1}^\infty \frac{\sin n}{n^2}.\) Which of the following must be true?
\romanlist{
\item
\(S\) converges absolutely.
\item 
\(S\) converges conditionally.
\item
\(S\) converges.
}
\choices
{I only}
{I and II only}
{\ans{I and III only}}
{II and III only}
{I, II, and III}
\end{question}

\begin{question}
Let \(f(x) = \sum_{n = 1}^\infty \frac{(-1)^n \, x^{2n + 1}}{n!}.\) 
What is the coefficient of the \(x^6\) term in \(\int_0^x f(t) \di t \; ?\)
\choicesline
{\(-\frac{7}{6}\)}
{\(-\frac{1}{12}\)}
{\(-\frac{1}{48}\)}
{\ans{\(\frac{1}{12}\)}}
{\(\frac{7}{6}\)}
\end{question}

\clearpage

\begin{question}
What is the interval of convergence of \(\sum_{n = 1}^\infty \frac{(-1)^n \, x^{2n + 1}}{2n + 1} \; ?\)
\choices
{The series diverges for all \(x.\)}
{\(-1 < x < 1\)}
{\ans{\(-1 \leqslant x \leqslant 1\)}}
{\(\nspace 0 < x < 1\)}
{\(\nspace 0 \leqslant x \leqslant 1\)}
\end{question}

\begin{question}
Let \(f(x) = \sum_{n = 1}^\infty \frac{(-1)^{n + 1} x^{n + 4} }{n^2}.\) The coefficient of the \(x^6\) term in \(f \, '(x)\) is 

\choicesline
{\(-\frac{7}{9}\)}
{\(-\frac{1}{4}\)}
{\(-\frac{1}{24}\)}
{\(\frac{1}{6}\)}
{\ans{\(\frac{7}{9}\)}}
\end{question}

\clearpage 

\begin{question}
Function \(g\) has derivatives of all orders. The fourth-degree Maclaurin series of \(g(x)\) is \(m(x).\) It is known that \(|g^{(5)}(x)| \leqslant 0.7\) for \(0 \leqslant x \leqslant 0.4.\) Which choice must be true?

\choices
{\(\left| g(0.4) - m(0.4) \right| \leqslant 0.7\)}
{\(\left| g(0.4) - m(0.4) \right| \leqslant \frac{0.7}{4!} (0.4)^4\)}
{\(\left| g(0.4) - m(0.4) \right| \geqslant \frac{0.7}{4!} (0.4)^4\)}
{\(\left| g(0.4) - m(0.4) \right| \geqslant \frac{0.7}{5!} (0.4)^5\)}
{\ans{\(\left| g(0.4) - m(0.4) \right| \leqslant \frac{0.7}{5!} (0.4)^5\)}}
\end{question}

\begin{question}
The power series \(S = \sum_{n = 0}^\infty c_n (x - 5)^n\) converges at \(x = 7\) and diverges at \(x = 8.\) Which of the following must be true?
\romanlist{
\item
\(S\) converges at \(x = 4.\)
\item
\(S\) converges at \(x = 3.\)
\item 
\(S\) diverges at \(x = 2.\)

}
\choices
{\ans{I only}}
{II only}
{III only}
{II and III only}
{I, II, and III}
\end{question}

\begin{question}
The power series \(f(x) = \sum_{n = 1}^\infty c_n (x - a)^n\) has a radius of convergence of \(R.\) The power series of which of the following must also have a radius of convergence of \(R \; ?\)
\romanlist{
\item
\(f \, '(x)\)
\item
\(\int f(x) \di x\)
\item
\(2f(x)\)
}

\choices
{None}
{I only}
{II only}
{\ans{I and II only}}
{I, II, and III}
\end{question}

\begin{question}
Let \(S = \sum_{n = 1}^\infty \frac{(-1)^n}{\sqrt{n + 8}}.\) Using the Alternating Series Error Bound, what is the least amount of terms must be summed to guarantee a partial sum that is within \(\frac{1}{30}\) of \(S \; ?\)
\choicesline
{22}
{30}
{\ans{891}}
{900}
{908}
\end{question}

\clearpage 

\begin{question}
\qspace{The partial sum \(S_k = \sum_{n = 1}^k \frac{1}{n^3}\) is used to estimate \(\sum_{n = 1}^\infty \frac{1}{n^3}.\)
Based on the Integral Test, what is the smallest value of \(k\) such that the error in \(S_k\) is no more than \(\frac{1}{200} \; ?\)
} 
\choicesline
{\(2\)}
{\ans{\(10\)}}
{\(50\)}
{\(100\)}
{\(200\)}
\end{question}

\begin{question}
\qspace{
Let \(f(x) = \frac{x}{1 + x^6}.\) Let \(g(x) = \int f(x) \di x\) and \(g(0) = 1.\) What is the Maclaurin series expansion of \(g \; ?\)}
\choices
{\(\frac{x^2}{2} - \frac{x^8}{8} + \frac{x^{14}}{14} - \frac{x^{20}}{20} + \cdots + \frac{(-1)^n \, x^{6n + 2}}{6n + 2} + \cdots\)}
{\(x - x^7 + x^{13} - x^{19} + \cdots + (-1)^n \, x^{6n + 1} + \cdots\)}  
{\(1 + x - x^7 + x^{13} + \cdots + (-1)^n \, x^{6n + 1} + \cdots\)}
{\ans{\(1 + \frac{x^2}{2} - \frac{x^8}{8} + \frac{x^{14}}{14} + \cdots + \frac{(-1)^n \, x^{6n + 2}}{6n + 2} + \cdots\)}} 
{\(1 - 7x^6 + 13x^{12} - 19 x^{18} + \cdots + (-1)^n (6n + 1) \, x^{6n} + \cdots\)}
\end{question}

\end{list}

\clearpage 

\twocolumn

\begin{supertabular}{ll}
% e.g., 1 & B \\
{\bf1}. & C \\ 
{\bf2}. & D \\ 
{\bf3}. & C \\ 
{\bf4}. & A \\ 
{\bf5}. & E \\ 
{\bf6}. & D \\ 
{\bf7}. & C \\ 
{\bf8}. & A \\ 
{\bf9}. & D \\ 
{\bf10}. & C \\ 
{\bf11}. & E \\ 
{\bf12}. & B \\ 
{\bf13}. & E \\ 
{\bf14}. & A \\ 
{\bf15}. & A \\ 
{\bf16}. & D \\ 
{\bf17}. & B \\ 
{\bf18}. & A \\ 
{\bf19}. & B \\ 
{\bf20}. & C \\ 
{\bf21}. & E \\ 
{\bf22}. & A \\ 
{\bf23}. & B \\ 
{\bf24}. & D \\ 
{\bf25}. & D \\ 
{\bf26}. & E \\ 
{\bf27}. & E \\ 
{\bf28}. & E \\ 
{\bf29}. & C \\ 
{\bf30}. & C \\ 
{\bf31}. & E \\ 
{\bf32}. & D \\ 
{\bf33}. & A \\ 
{\bf34}. & E \\ 
{\bf35}. & B \\ 
{\bf36}. & C \\ 
{\bf37}. & D \\ 
{\bf38}. & C \\ 
{\bf39}. & E \\ 
{\bf40}. & E \\ 
{\bf41}. & A \\ 
{\bf42}. & D \\ 
{\bf43}. & C \\ 
{\bf44}. & B \\ 
{\bf45}. & D \\ 

\end{supertabular}
\end{document}