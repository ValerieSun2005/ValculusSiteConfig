\begin{question}
If \(\lim_{x \to 3} f(x) = 8,\) then which option is true?
\choices
{\(f(3) = 8.\)}
{\ans{As \(x\) approaches \(3,\) \(f(x)\) approaches \(8.\)}}
{\(f(8) = 3.\)} 
{As \(x\) approaches \(8,\) \(f(x)\) approaches \(3.\)} 
{\(f\) is continuous at \(x = 3.\)}
\end{question}

\begin{question}
\(\lim_{x \to 0} \frac{\sin 3x}{x}\) is 
\choicesline
{\(0\)}
{\(\frac{1}{3}\)}
{1}
{\ans 3}
{nonexistent}
\end{question}

\begin{question}
\(\lim_{x \to -7} \frac{x + 7}{49 - x^2}\) is 
\choicesline
{\(-7\)}
{\ans{\(-\frac{1}{14}\)}}
{\(-\frac{1}{49}\)}
{\(\frac{1}{14}\)}
{\(49\)}
\end{question}

\begin{question}
\(\lim_{x \to 0} \frac{x}{\tan x}\) is 
\choicesline
{\(-1\)}
{0}
{\ans 1}
{\(\pi\)}
{nonexistent}
\end{question}

\begin{question}
\(\lim_{x \to \infty} \frac{3x^4 - 8x^3 + x^2 - 10}{2x^2 - 5x^3 - 10x^4}\) is 
\choicesline
{\ans{\(-\frac{3}{10}\)}}
{\(0\)}
{\(\frac{3}{10}\)}
{\(\frac{3}{2}\)}
{nonexistent}
\end{question}

\begin{question}
The horizontal asymptote of \(f(x) = \frac{3x^3 + x^2 - 4}{8 - 5x^3}\) \, is \\
\choicesline
{\(y = -\frac{5}{3}\)}
{\ans{\(y = -\frac{3}{5}\)}}
{\(y = \frac{3}{8}\)}
{\(y = \frac{3}{5}\)}
{\(y = \frac{5}{3}\)}
\end{question}

\begin{question}
\(\lim_{x \to \pi/2} \tan 2x\) is
\choicesline
{\(-1\)}
{\(-\frac{1}{2}\)}
{\ans 0}
{1}
{\(\pi\)}
\end{question}

\begin{question}
Function \(g\) is discontinuous at \(x = 5.\)
Selected values of $g$ are shown in the table below. \\[1ex]
\renewcommand{\arraystretch}{1.2}
\begin{center}
\begin{tabularx}{0.7\textwidth} { 
  | >{\arraybackslash}X 
  || >{\centering\arraybackslash}X 
  | >{\centering\arraybackslash}X 
  | >{\centering\arraybackslash}X 
  | >{\centering\arraybackslash}X 
  | >{\centering\arraybackslash}X | }
 \hline
 $\hspace{2.2ex} x$ & 4.99 & 4.999 & 5 & 5.0001 &5.001  \\
 \hline
 $f(x)$ & 2.99 & 2.999 & $-4$ & 3.001  &3.01 \\
 \hline
\end{tabularx} \\[1.5ex]
\end{center}
A reasonable estimate for \(\lim_{x \to 5} g(x)\) is \\
\choicesline
{\(-4\)}
{\(-3\)}
{2}
{\ans 3}
{5}
\end{question}

\begin{question}
If
\(f(x)= \begin{cases} 
3x - 2 \cos x& x < \pi \\
x^2 & x \geqslant \pi, \\ 
\end{cases} 
\) \, 
then \(\lim_{x \to \pi^-} f(x)\) is \\
\choicesline
{\(-\pi^2\)}
{\(3 \pi\)}
{\(3 \pi - 2\)}
{\ans{\(3 \pi + 2\)}}
{\(\pi^2\)}
\end{question}

\begin{question}
\(\lim_{x \to 3} \frac{\sqrt{x - 2} - 1}{9 - 3x}\) is 
\choicesline
{\(-\frac{1}{2}\)}
{\ans{\(-\frac{1}{6}\)}}
{\(\frac{1}{6}\)}
{\(\frac{1}{2}\)}
{nonexistent}
\end{question}

\begin{question}
Given that \(\lim_{x \to a} f(x)\) exists, which statements must be true?
\romanlist{
\item
\(f(x)\) is continuous at \(x = a.\)
\item
\(\lim_{x \to a^-} f(x) = \lim_{x \to a^+} f(x).\)
\item
\(f(a)\) is defined.
}
\choices
{I only}
{\ans{II only}}
{I and II only}
{II and III only}
{I, II, and III}
\end{question}

\begin{question}
If \(\lim_{k \to 0} \frac{e^k - 1}{k} = 1,\) then \(\lim_{k \to 0} \frac{e^2 - e^{2 + k}}{k}\) is \\
\choicesline
{\ans{\(-e^2\)}}
{\(-e^{-2}\)}
{\(e^{-2}\)}
{1}
{\(e^2\)}
\end{question}

\begin{question}
\(g(x) = \frac{x^2 - 5x + 6}{x - 3}\) \; has a removable discontinuity at \\ 
\choicesline
{\(x = -3\)}
{\(x = -2\)}
{\(x = 0\)}
{\(x = 2\)}
{\ans{\(x = 3\)}}
\end{question}

\begin{question}
\(\lim_{x \to \infty} \frac{\sin(2x)}{x + 2}\) is 
\choicesline
{\(-1\)}
{\ans 0}
{1}
{2}
{\(\infty\)}
\end{question}

\begin{question}
Let
\( f(x) = \begin{cases} 
2 - kx & x\leqslant 3 \\
kx^2 -22 & x > 3. \\ 
\end{cases}
\)
For what value of \(k\) is \(f\) continuous at \(x = 3 \; ?\) \\
\choicesline
{\(-4\)}
{0}
{1}
{\ans 2}
{3}
\end{question}

\begin{question}
The oblique asymptote of \(f(x) = \frac{x^2 + 7x + 1}{x - 2}\) is \\ 
\choicesline
{\(y = 9\)}
{\(y = x - 9\)}
{\(y = x\)}
{\(y = x + 5\)}
{\ans{\(y = x + 9\)}}
\end{question}

\begin{question}
\(\lim_{x \to \infty} \frac{\sqrt{4x^2 - 1}}{x + 3} \, \) is 
\choicesline
{\(-4\)}
{\(-2\)}
{\ans 2}
{4}
{nonexistent}
\end{question}

\begin{minipage}{\textwidth}
Questions \ref{q:graph-start}--\ref{q:graph-end} refer to the following graph.
%
\begin{center}
\begin{tikzpicture}
\begin{axis}[
    axis lines = middle,
    xlabel=\large $x$,
    ylabel=\large $y$,
    every axis x label/.style={at={(current axis.right of origin)},anchor=west},
    every axis y label/.style={at=(current axis.above origin),anchor=south},
    xmin = -5.5, xmax = 5.5,
    ymin = -5.5, ymax = 5.5,
    xticklabel={
        % test if the x value is below zero
        \ifdim \tick pt < 0pt
          % if yes, calculate the absolute value
          \pgfmathparse{abs(\tick)}
          % and print first a minus sign in a zero-width box, followed by the absolute value
          \llap{$-{}$}\pgfmathprintnumber{\pgfmathresult}
      \else
         % if no, print as usual
          \pgfmathprintnumber{\tick}
      \fi
        },
    xtick distance = 1,
    ytick distance = 1,
    grid = both,
    major grid style = {lightgray},
    minor grid style = {lightgray!25},
    width = 0.7 \textwidth,
    height = 0.7 \textwidth]
    \addplot[red, ultra thick, smooth, domain=-5:0]{x + 4};
    \addplot[red, ultra thick, smooth, domain=0:4.8]{2 - x}node[pos=1.07]{\Large $f$};
    \addplot[agua!70, ultra thick, smooth, domain=-4.8:0]{(x + 2)^2 - 3};
    \addplot[agua!70, ultra thick, smooth, domain=0:4.8]{2 + (1/2)* sin(deg(x))}node[pos=1.07]{\Large $g$};
    \plotCircle{(0, 2)}{red}{red}{3};
    \plotCircle{(0, 1)}{agua!70}{agua!70}{3};
    \plotCircle{(0, 4)}{red}{white}{3};
\end{axis}
\end{tikzpicture}
\end{center}
\end{minipage}

\begin{question}
\label{q:graph-start}
\(\lim_{x \to 0} f(x)\) is 
\choicesline
{0}
{1}
{2}
{4}
{\ans{nonexistent}}
\end{question}

\begin{question}
\(\lim_{x \to 0^+} g(x)\) is
\choicesline
{0}
{1}
{\ans 2}
{4}
{nonexistent}
\end{question}

\begin{question}
\(\lim_{x \to -2} [f(x) - g(x)]\) is \\
\choicesline
{\(-5\)}
{\(-1\)}
{1}
{\ans 5}
{nonexistent}
\end{question}

\begin{question}
\(\lim_{x \to -2} [g(x)]^2\) is 
\choicesline
{\(-4\)}
{\(-2\)}
{2}
{4}
{\ans 9}
\end{question}

\begin{question}
\(\lim_{x \to 0} [f(x) g(x)]\) is 
\choicesline
{1}
{2}
{\ans 4}
{8}
{nonexistent}
\end{question}

\begin{question}
\label{q:graph-end}
\(\lim_{x \to 0} f(-x^2)\) is 
\choicesline
{0}
{1}
{2}
{\ans 4}
{nonexistent}
\end{question}

\begin{question}
On what interval of \(x\) is \(f(x) = \frac{\ln(x - 2)}{x^2 - 9}\) continuous?
\choices
{\((-\infty, -3)\cup(3, \infty)\)}
{\((-\infty, -2)\)}
{\((2, \infty)\)}
{\((3, \infty)\)}
{\ans{\((2, 3)\cup(3, \infty)\)}}
\end{question}

\begin{question}
\(\lim_{x \to -3} \frac{5\sin(x + 3)}{6 + 2x}\) is 
\choicesline
{0}
{\(\frac{5}{6}\)}
{1}
{\ans{\(\frac{5}{2}\)}}
{nonexistent}
\end{question}

\begin{question}
Function \(f\) is continuous and satisfies \(f(4) = 8.\) If \(\lim_{x \to 2} g(x) = 4,\) then \(\lim_{x \to 2} f(g(x))\) is \\
\choicesline
{\(-8\)}
{\(-4\)}
{2}
{4}
{\ans 8}
\end{question}

\begin{question}
\(\lim_{x \to 2} \cfrac{\cfrac{1}{2 \Bstrut} - \cfrac{1}{x \Bstrut} }{\Tstrut 2 - x}\) is
\choicesline
{\(-\frac{1}{2}\)}
{\ans{\(-\frac{1}{4}\)}}
{\(\frac{1}{4}\)}
{\(\frac{1}{2}\)}
{nonexistent}
\end{question}

\begin{question}
\(\lim_{x \to 0} x \sin\left( \frac{1}{x^2}\right)\) is 
\choicesline
{\ans 0}
{\(\frac{1}{4}\)}
{\(\frac{1}{2}\)}
{1}
{nonexistent}
\end{question}

\begin{question}
Let
\(f(x)= \begin{cases} 
x^2 + 1& x < p \\
2x & x \geqslant p. \\ 
\end{cases} 
\)
If \(f(x)\) is continuous at \(x = p,\) then \(p\) is \\
\choicesline
{\ans{\(-1\)}}
{0}
{1}
{2}
{4}
\end{question}

\begin{question}
\(\lim_{t \to \infty} \sin t\) is 
\choicesline
{\(-1\)}
{0}
{1}
{\(\pi\)}
{\ans{nonexistent}}
\end{question}

\begin{question}If \(\lim_{x \to 2} f(x) = -3,\)
then \(\lim_{x \to 2} \left([f(x)]^2 - 2x \right)\) is \\
\choicesline
{\(-7\)}
{\(-3\)}
{4}
{\ans 5}
{9}
\end{question}

\begin{question}
Function \(f\) is continuous. Selected values of \(f(x)\) are shown in the table below. \\
\begin{center}
\begin{tabularx}{0.7\textwidth} { 
  | >{\arraybackslash}X 
  || >{\centering\arraybackslash}X 
  | >{\centering\arraybackslash}X 
  | >{\centering\arraybackslash}X 
  | >{\centering\arraybackslash}X 
  | >{\centering\arraybackslash}X | }
 \hline
 \(\hspace{2.2ex} x\) & \(-1\) & 2 & 3 & 6 & 11  \\
 \hline
 \(f(x)\) & 2 & 1 & \(1\) & 1  & 2 \\
 \hline
\end{tabularx} \\
\end{center}
Following the Intermediate Value Theorem, which value of \(f(x)\) is guaranteed to exist for \(-1 \leqslant x \leqslant 11 \; ?\) \\
\choicesline
{\ans 0}
{3}
{5}
{6}
{11}
\end{question}

\begin{question}
\(\lim_{x \to -\infty} \frac{\sqrt{4x^6 - 4x^2 + 1}}{3x^3 + 2} \, \) is 
\choicesline
{\ans{\(-\infty\)}}
{\(-\frac{2}{3}\)}
{\(\frac{2}{3}\)}
{\(\infty\)}
{nonexistent}
\end{question}

\begin{question}
\(\lim_{x \to \infty} x \sin \left(\frac{1}{x}\right)\) is 
\choicesline
{\(-1\)}
{0}
{\ans 1}
{\(\pi\)}
{nonexistent}
\end{question}

\begin{question}
At \(x = 4,\) which choice about \(g(x) = \frac{12 + x - x^2}{x - 4}\) is true?
\choices
{\(g(x)\) has a vertical asymptote at \(x = 4.\)}
{\(g(x)\) has a jump discontinuity at \(x = 4.\)}
{\ans{\(g(x)\) has a removeable discontinuity at \(x = 4.\)}}
{\(\lim_{x \to 4} g(x)\) does not exist.}
{\(g(x)\) is continuous at \(x = 4.\)}
\end{question}

\begin{question}
\(\lim_{x \to 0^+} \ln(\sin x)\) is 
\choicesline
{\ans{\(-\infty\)}}
{0}
{1}
{\(e\)}
{\(\infty\)}
\end{question}

\begin{question}
\(\lim_{x \to 0} \frac{x^2}{\sin^2 2x}\) is 
\choicesline
{0}
{\ans{\(\frac{1}{4}\)}}
{1}
{4}
{nonexistent}
\end{question}

\begin{question}
\(\lim_{x \to 0} \frac{x - x\cos(x)}{x^2}\) is 
\choicesline
{\(-\pi\)}
{\ans 0}
{1}
{\(\pi\)}
{nonexistent}
\end{question}

\begin{question}
\(\lim_{x \to 0^+} \arctan \left(\frac{1}{x}\right)\) is 
\choicesline
{\(-\infty\)}
{\(-\frac{\pi}{2}\)}
{0}
{\ans{\(\frac{\pi}{2}\)}}
{\(\infty\)}
\end{question}

\begin{question}
\(\lim_{x \to \infty} \frac{5e^{x} - x}{8e^{x} + 9}\) is 
\choicesline
{\(-\frac{5}{8}\)}
{0}
{\ans{\(\frac{5}{8}\)}}
{1}
{\(\infty\)}
\end{question}

\begin{question}
\(\lim_{x \to \pi/4} \, \frac{\cos 2x}{\cos x - \sin x}\) is 
\choicesline
{0}
{\(\frac{\sqrt 2}{2}\)}
{\ans{\(\sqrt 2\)}}
{\(\pi\)}
{nonexistent}
\end{question}

\begin{question}
Functions \(g\) and \(h\) are continuous and satisfy \(g(1) = h(1) = 3.\) Function \(f\) satisfies \(g(x) \leqslant f(x) \leqslant h(x)\) for \(0 \leqslant x \leqslant 2.\) 
Which statements must be true?
\romanlist{
\item
\(\lim_{x \to 1} g(x) = \lim_{x \to 1} h(x) = 3.\) 
\item
\(\lim_{x \to 1} f(x) = 3.\)
\item
\(f(x)\) is continuous at \(x = 1.\)
}
\choices
{I only}
{II only}
{I and II only}
{II and III only}
{\ans{I, II, and III}}
\end{question}

\begin{question}
If \(f(1) = 2,\) \(\lim_{x \to 1^-} f(x) = 4,\) and \(\lim_{x \to 1^+} f(x) = -1,\) then \(\lim_{x \to 1} f(\cos(x - 1))\) is 
\choicesline
{\(-1\)}
{0}
{2}
{\ans 4}
{nonexistent}
\end{question}

\begin{question}
If \(\lim_{x \to 3} (2x + 4) = 10,\) then \(|(2x + 4) - 10| < \epsilon\) and \(|x - 3| < \delta,\) where \(\delta =\) \\
\choicesline
{\(\frac{\epsilon}{4}\)}
{\ans{\(\frac{\epsilon}{2}\)}}
{\(\epsilon\)}
{\(2 \epsilon\)}
{\(4 \epsilon\)}
\end{question}

\begin{question}
If \(\lim_{x \to a} f(x) = \infty,\) then which option is true?
\choices
{For positive \(M,\) there exists a positive \(\delta\) such that \(f(x) > M\) for \(|x - a| > 0.\)}
{For positive \(M,\) there exists a positive \(\delta\) such that \(f(x) < M\) for \(|x - a| > \delta.\)}
{For positive \(M,\) there exists a positive \(\delta\) such that \(f(x) > M\) for \(|x - a| > \delta.\)}
{For positive \(M,\) there exists a positive \(\delta\) such that \(f(x) < M\) for \(0 < |x - a| < \delta.\)}
{\ans{For positive \(M,\) there exists a positive \(\delta\) such that \(f(x) > M\) for \(0 < |x - a| < \delta.\)}}
\end{question}
