\begin{question}
A line connects the points \((0, 1)\) and \((1, 5).\) Its slope is \\
\choicesline
{\(-4\)}
{\(-1\)}
{1}
{\ans 4}
{5}
\end{question}

\begin{question}
The function \(ax^2 + 2x + c\) is rewritten in the form \(a(x - h)^2 + k,\) where \(h\) and \(k\) are constants. What is the value of \(k \; ?\) \\
\choicesline
{\(\frac{1}{2a^2}\)}
{\(\frac{1}{a^2}\)}
{\ans{\(\frac{ac - 1}{a}\)}}
{\(\frac{a^2 c - 1}{a}\)}
{\(\frac{1 - ac}{a}\)}
\end{question}

\begin{question}
\(\frac{x - 2}{x + 4} > 0\) if \\
\choices
{\(x < -4\)}
{\(-4 < x < 2\)}
{\(-4 < x \leqslant 2\)}
{\(x > 2\)}
{\ans{\(x < -4, \, x > 2\)}}    
\end{question}

\begin{question} 
The domain of \(e^{\sqrt{x - 2}} - 6\) is \\
\choices
{\(x \geqslant -6\)}
{\(x \geqslant 0\)}
{\(x \leqslant 2\)}
{\ans{\(x \geqslant 2\)}}
{all real numbers}
\end{question} 

\begin{question} 
For \(\pi < \theta < \frac{3 \pi}{2},\) \(\cos \theta = -\frac{1}{5}.\) Then \(\sin \theta = \) \\
\choicesline
{\ans{\(-\frac{\sqrt{24}}{5}\)}}
{\(-\frac{4}{5}\)}
{\(\frac{1}{5}\)}
{\(\frac{4}{5}\)}
{\(\frac{\sqrt{24}}{5}\)}
\end{question} 

\begin{question}
What is the solution set of \(x\) in the following system of equations?
\[
\begin{aligned}
    y &= 5x - 3 \\
    y &= x^2 + 1
\end{aligned}
\]
\choices
{\(\{-17, -2\}\)}
{\(\{-4, -1\}\)}
{\ans{\(\{1, 4\}\)}}
{\(\{2, 17\}\)}
{\(\{4\}\)}
\end{question}

\begin{question}
If \(f(x) = x^3 - 4,\) then \(f^{-1}(x) =\)
\choices
{\ans{\(f^{-1}(x) = \sqrt[3]{x + 4}\)}}
{\(f^{-1}(x) = \frac{1}{x^3 - 4}\)}
{\(f^{-1}(x) = \sqrt[3]{x - 4}\)}
{\(f^{-1}(x) = \sqrt[3]{x^3 - 8}\)}
{\(f^{-1}(x) = \frac{1}{\sqrt[3]{x + 4}}\)}
\end{question}

\begin{question}
A man stands 50 feet from a building that is 200 feet tall. The angle between the man and the top of the building is \\
\choicesline
{\ans{\(\tan^{-1}(4)\)}}
{\(\tan^{-1} \left(\frac{1}{4}\right)\)}
{\(\sin^{-1}(4)\)}
{\(\sin^{-1} \left(\frac{1}{4}\right)\)}
{\(\cos^{-1}(4)\)}
\end{question}

\begin{question}
The horizontal asymptote of \(f(x) = \frac{2x^3 + 8x^2 - 6x}{4 + 8x^3} + 1\) is \\
\choicesline
{\(y = 0\)}
{\(y = \frac{1}{4}\)}
{\(y = 1\)}
{\ans{\(y = \frac{5}{4}\)}}
{\(y = 2\)}
\end{question}

\begin{question} 
Which transformation must be performed to \(f(x)\) to obtain the new function \(g(x) = 3f(2x - 4) + 7 \; ?\)
\choices
{\(f(x)\) must be shifted \(2\) units to the left.}
{\ans{\(f(x)\) must be shifted \(2\) units to the right.}}
{\(f(x)\) must be shifted \(4\) units to the left.}
{\(f(x)\) must be shifted \(4\) units to the right.}
{\(f(x)\) must be shifted down \(7\) units.}
\end{question}

\begin{question}
\(\cos \left(\frac{3 \pi}{4}\right)\) is 
\choicesline
{\(-\frac{\sqrt 3}{2}\)}
{\ans{\(-\frac{\sqrt 2}{2}\)}}
{\(-\frac{1}{2}\)}
{\(\frac{1}{2}\)}
{\(\frac{\sqrt 2}{2}\)}
\end{question}

\begin{question}
The slope of \(x - 2y = 3\) is \\
\choicesline
{\(-2\)}
{\(-1\)}
{\(-\frac{1}{2}\)}
{\ans{\(\frac{1}{2}\)}}
{\(1\)}
\end{question}

\begin{question}
Which statement is true about \(f(x) = x^2 - 4x + 5 \; ?\)
\choices
{\(f(x)\) has a minimum at \((-2, 1).\)}
{\(f(x)\) has a maximum at \((-2, 1).\)}
{\ans{\(f(x)\) has a minimum at \((2, 1).\)}}
{\(f(x)\) has a maximum at \((2, 1).\)}
{\(f(x)\) has a minimum at \((4, 5).\)}
\end{question}

\begin{question}
Which function has a vertical asymptote at \(x = 1 \; ?\)
\choices
{\ans{\(f(x) = \frac{1}{x^2 - 3x + 2}\)}}
{\(f(x) = \sqrt{x - 1}\)}
{\(f(x) = \frac{x^2 - 1}{x - 1}\)}
{\(f(x) = \frac{x - 1}{x + 1}\)}
{\(f(x) = \frac{x^2}{x}\)}
\end{question}

\begin{question}
Which expression is equivalent to \(\left(4w^2 x^8 y^{-2} \sqrt z \right)^{-1/2} \; ?\) \\
\choicesline
{\(\frac{2w x^4 \sqrt[4]{z}}{y}\)}
{\ans{\(\frac{y}{2wx^4 \sqrt[4]{z}}\)}}
{\(\frac{2y}{wx^4 \sqrt[4] z}\)}
{\(\frac{wx^4 \sqrt[4] z}{2y}\)}
{\(\frac{2wx^4}{y^2 \sqrt z}\)}
\end{question}

\begin{question}
\(\frac{2x^2 - 5x + 9}{x - 1} = \)
\choices
{\(2x^2 - 3x + 6\)}
{\ans{\(2x - 3 + \frac{6}{x - 1}\)}}
{\(2x^2 - 3x + \frac{6}{x - 1}\)}
{\(2x - 3\)}
{\(2x - 7 + \frac{2}{x - 1}\)}
\end{question}

\begin{question}
Assuming that the domains are restricted to avoid division by zero,    \(\frac{x^2 - 10x + 24}{x - 3} \cdot \frac{2(x - 6)^{-1}}{x - 4} = \)
\choices
{\(\frac{2(x - 4)}{x - 3}\)}
{\ans{\(\frac{2}{x - 3}\)}}
{\(\frac{2}{(x - 3)(x - 6)}\)}
{\(\frac{2(x - 6)(x - 4)}{x - 3}\)}
{\(\frac{2}{x - 4}\)}
\end{question}

\begin{question}
The domain of \(g(x) = \frac{\log_2(x)}{x - 5}\) is 
\choices
{\(x \ne 0\)}
{\(x > 0\)}
{\(x \ne 5\)}
{\ans{\(x > 0, \, x \ne 5\)}}
{all real numbers}
\end{question}

\begin{question}
Which option best describes the end behavior of \(f(x) = e^{-2x} - 2 \; ?\)
\choices
{As \(x \to \infty, f(x) \to -\infty.\)}
{\ans{As \(x \to \infty, f(x) \to -2.\)}}
{As \(x \to \infty, f(x) \to 0.\)}
{As \(x \to \infty, f(x) \to 1.\)}
{As \(x \to \infty, f(x) \to \infty.\)}
\end{question}

\begin{question}
\(\frac{\sin^8(x) - \cos^8(x)}{\sin^4(x) + \cos^4(x)} =\)
\choices
{\(\frac{1}{2}\)}
{\(1\)}
{\ans{\(\sin^2(x) - \cos^2(x)\)}}
{\(\sin^4(x) + \cos^4(x)\)}
{\(\cos^2(x) - \sin^2(x)\)}
\end{question}

\begin{question}
The period of \(7 \sin \left(3x - \frac{\pi}{4}\right) - 2\) is \\
\choicesline
{\(\frac{\pi}{3}\)}
{\ans{\(\frac{2 \pi}{3}\)}}
{\(\pi\)}
{\(2 \pi\)}
{\(4 \pi \)}
\end{question}

\begin{question}
\(\log(12) =\) \\
\choices
{\(2 \log 6\)}
{\(3 \log 4\)}
{\ans{\(\log(4) + \log(3)\)}}
{\(\log(12) \log(1)\)}
{\(\log(6) \log(2)\)}
\end{question}

\begin{question}
If \(a = be^{cd},\) then \(d =\)
\choices
{\(\frac{\ln(b) - \ln(a)}{c}\)}
{\(\frac{\ln a}{c \ln b}\)}
{\(\sqrt[c]{\frac{a}{b}}\)}
{\(\frac{1}{c} e^{a/b}\)}
{\ans{\(\frac{\ln(a) - \ln(b)}{c}\)}}
\end{question}

\begin{question}
\(\sqrt a + \sqrt b = \)
\choices
{\(\sqrt{a + b}\)}
{\(\sqrt{a - b}\)}
{\(\sqrt{ab}\)}
{\(a \sqrt{b}\)}
{\ans{None of the above}}
\end{question}

\begin{question} 
What are the solutions to the equation \(\frac{x^2 - 5x - 24}{x - 8} = 2x -5 \; ?\) \\ 
\choices
{\(x = -3\)}
{\(x = \frac{5}{2}\)}
{\(x = 8\)}
{\(x = -3, x = 8\)}
{\ans{The equation has no solution.}}
\end{question}

\begin{question} 
\(4^a \, \frac{2^{3b - 1}}{8^{1 - c}} =\)
\choices
{\(2^{a + 3b + c - 2}\)}
{\(2^{a + 3b - c}\)}
{\(2^{2a + 3b - 3c + 2}\)}
{\ans{\(2^{2a + 3b + 3c - 4}\)}}
{\(2^{a(3b - 1)/(1 - c)}\)}
\end{question}

\begin{question}
Which function does \emph{not} have all real numbers as its range?
\choices
{\(y = 3x\)}
{\(y = 4x^3 - 8x^2 + x - 17\)}
{\(y = \log_4(x + 2)\)}
{\ans{\(y = \frac{x^2 - 3x - 10}{x + 2}\)}}
{\(y = \frac{e^x}{x - 3}\)}
\end{question}

\begin{question}
What values of \(x\) satisfy \(\ln(x - 3) + \ln(x + 1) = \ln(5) \; ?\) 
\choices
{\(x = -4, \, x = 2\)}
{\(x = -2, \, x = 4\)}
{\(x = -2\)}
{\(x = 2, \, x = 4\)}
{\ans{\(x = 4\)}}
\end{question}

\begin{question}
What is an equation of the line that passes through the point \((-2, 1)\) and has a slope of \(-\frac{1}{2} \; ?\)
\choices
{\(y + 1 = -\frac{1}{2}(x - 2)\)}
{\(y + 1 = -\frac{1}{2}(x + 2)\)}
{\ans{\(y - 1 = -\frac{1}{2}(x + 2)\)}}
{{\(y + 1 = \frac{1}{2}(x - 2)\)}}
{{\(y - 1 = \frac{1}{2}(x + 2)\)}}
\end{question}

\begin{question}
The solution set of \(x\) in \(\frac{3}{x - 1} = \frac{x - 2}{2}\) is \\
\choicesline
{\(\{-4, 1\}\)}
{\(\left\{-\frac{3}{2}, 1\right\}\)}
{\ans{\(\{-1, 4\}\)}}
{\(\{-1\}\)}
{\(\{4\}\)}
\end{question}

\begin{question}
The range of \(-3 \cos \left(4x - \frac{\pi}{3}\right) + 7\) is \\
\choicesline
{\([-4, 10]\)}
{\([-3, 3]\)}
{\ans{\([4, 10]\)}}
{\([5, 11]\)}
{\([11, 15]\)}
\end{question}

\begin{question}
A downward-opening parabola intersects the \(y\)-axis at \((0, 30)\) and has zeros of \(x = -2\) and \(x = 5.\) The parabola's equation is 
\choices
{\(y = (x - 2)(x - 5)\)}
{\(y = (x + 2)(x + 5)\)}
{\(y = 3(x - 2)(x - 5)\)}
{\(y = -3(x + 2)(x + 5)\)}
{\ans{\(y = -3(x - 2)(x - 5)\)}}
\end{question}

\begin{question}
\(\cos \left(\tan^{-1} x\right) =\) \\
\choicesline
{\(\frac{x}{\sqrt{1 + x^2}}\)}
{\(x\)}
{\ans{\(\frac{1}{\sqrt{1 + x^2}}\)}}
{\(\frac{1}{\sqrt{1 - x^2}}\)}
{\(\frac{\sqrt{1 + x^2}}{x}\)}
\end{question}

\begin{question}
Which function does \emph{not} have all real  numbers as its domain?
\choices
{\ans{\(y = \frac{x^2}{x}\)}}
{\(y = 2^x\)}
{\(y = -3 \sin(x - 4) + 9\)}
{\(y = |x - 4| - 2\)}
{\(y = -7\)}
\end{question}

\begin{question}
If \(f(x) = x^2 - 4\) and \(g(x) = \sin^2(x - 3),\) then \(g(f(x)) =\) \\ 
\choices
{\(\sin^4(x - 3) - 4\)}
{\ans{\(\sin^2(x^2 - 7)\)}}
{\(\sin^4(x - 3) - 4 \sin^2(x - 3)\)}
{\(\sin^2(x^2 - x - 7)\)}
{\(\sin^2(x^2 - 4)\)}
\end{question}

\begin{question}
What value of \(x\) satisfies \(2^{3x - 1} = 8^{4x - 2} \; ?\) \\
\choicesline
{\(\log_2 \left(\frac{1}{5}\right)\)}
{\(\log_2 \left(\frac{5}{9}\right)\)}
{\(\frac{1}{5}\)}
{\ans{\(\frac{5}{9}\)}}
{1}
\end{question}

\begin{question}
An object's kinetic energy varies directly as the square of the object's speed. A cannonball traveling at 5 meters per second has a kinetic energy of 100 joules. How much kinetic energy would the cannonball have if it traveled at 10 meters per second?
\choicesline
{40 joules}
{100 joules}
{200 joules}
{\ans{400 joules}}
{2000 joules}
\end{question}

\begin{question}
Polynomial \(f(x)\) contains the distinct factor \((x - p)^2.\)
Which statement describes the behavior of \(f(x)\) at \(x = p \; ?\)
\choices
{\(f(x)\) intersects the \(x\)-axis at \(x = p.\)}
{\(f(x)\) curves through the \(x\)-axis at \(x = p.\)}
{\ans{\(f(x)\) bounces off the \(x\)-axis at \(x = p.\)}}
{\(f(x)\) does not touch the \(x\)-axis at \(x = p.\)}
{\(f(x)\) is undefined at \(x = p.\)}
\end{question}

\begin{question}
\(f(x) = \log_3(\sqrt[3]{x - p})\) is undefined for \\
\choicesline
{\(p \leqslant 0\)}
{\(p \geqslant 0\)}
{\(p = x\)}
{\(p \leqslant x\)}
{\ans{\(p \geqslant x\)}}
\end{question}

\begin{question}
Which statement describes the end behavior of \(f(x) = x^6 + 2x^5 - x^2 + 4 \; ?\)
\choices
{As \(x \to -\infty,\) \(f(x) \to 0;\) as \(x \to \infty,\) \(f(x) \to 0.\)}
{As \(x \to -\infty,\) \(f(x) \to -\infty;\) as \(x \to \infty,\) \(f(x) \to -\infty.\)}
{As \(x \to -\infty,\) \(f(x) \to -\infty;\) as \(x \to \infty,\) \(f(x) \to \infty.\)}
{As \(x \to -\infty,\) \(f(x) \to \infty;\) as \(x \to \infty,\) \(f(x) -\to \infty.\)}
{\ans{As \(x \to -\infty,\) \(f(x) \to \infty;\) as \(x \to \infty,\) \(f(x) \to \infty.\)}}
\end{question}

\begin{question}
\(\sum_{i = 1}^5 (2i - 1) =\)

\choicesline
{8}
{20}
{\ans{25}}
{30}
{125}
\end{question}

\begin{question}
The range of \(\frac{2}{5 - x} - 1\) is 
\choices
{\ans{all real numbers}}
{\(x > -1\)}
{\(x > -\frac{3}{5}\)}
{\(x \ne 0\)}
{\(x \ne 5\)}
\end{question}

\begin{question}
What are the solutions to \(|x - 4| + 2 = 5 \; ?\)
\choices
{\(x = -1\)}
{\(x = 3\)}
{\(x = 7\)}
{\(x = -1, \, x = 7\)}
{\ans{\(x = 1, \, x = 7\)}}
\end{question}

\begin{question}
\(\sec \left(\frac{\pi}{2} \right)\) is 
\choicesline
{\(-1\)}
{0}
{1}
{\(\pi\)}
{\ans{undefined}}
\end{question}

\begin{question}
Which expression is equivalent to \(\frac{1}{a} + \frac{1}{b} \; ?\) \\
\choicesline
{\ans{\(\frac{a + b}{ab}\)}}
{\(\frac{1}{a + b}\)}
{\(\frac{1}{ab}\)}
{\(\frac{ab}{a + b}\)}
{\(\frac{2}{a + b}\)}
\end{question}

\begin{question}
Which expression is equivalent to \(\left(4a^4 b^3 c^{-7} d^{-5}\right)^2 \; ?\)
\choices
{\(16 a^8 b^6 c^{-14} d^{-10}\)}
{\(\frac{a^4 b^3}{16 c^7 d^5}\)}
{\ans{\(\frac{c^{14}}{16 a^8 b^6}\)}}
{\(\frac{a^2 b}{16 c^9 d^7}\)}
{\(\frac{16 c^9 d^7}{a^2 b}\)}
\end{question}

\begin{question}
A sphere with radius \(r\) has a volume of \(\frac{4}{3} \pi r^3\) and a surface area of \(4 \pi r^2.\)
A spherical ball has a surface area of \(400 \pi\) square meters. Its volume, in cubic meters, is 
\choicesline
{\(10\)}
{\(\frac{40 \pi}{3}\)}
{\(1000\)}
{\(\frac{400 \pi}{3}\)}
{\ans{\(\frac{4000 \pi}{3}\)}}
\end{question}

\begin{question}
\(\sin^{-1} \left(\sin \frac{2 \pi}{3}\right) =\) \\
\choicesline
{\(\frac{\pi}{6}\)}
{\ans{\(\frac{\pi}{3}\)}}
{\(\frac{2 \pi}{3}\)}
{\(\frac{1}{2}\)}
{\(\frac{\sqrt 3}{2}\)}
\end{question}

\begin{question}
Which expression is equivalent to \(e^{f(x)} \; ?\) (Note: \(\log x = \log_{10} x.\)) \\
\choices
{\ans{\(10^{f(x) \log e}\)}}
{\(10^{f(x) + \log e}\)}
{\(10^{f(x) \ln 10}\)}
{\(10^{f(x) - \log e}\)}
{\(10^{f(x) + \ln 10}\)}
\end{question}

\begin{question}
\(2x^2 - kx + 8\) has no real solutions for 
\choices
{\(-8 < k, \, k > 8\)}
{\(k < - \sqrt 2, \, k > \sqrt 2\)}
{\ans{\(-8 < k < 8\)}}
{\(-2 \sqrt 2 < k < 2 \sqrt 2\)}
{\(k < 64\)}
\end{question}

\begin{question}
If \(f(x) = x^2 + 4,\) then 
\(\frac{f(x + h) - f(x)}{h} =\) \\
\choicesline
{\ans{\(2x + h\)}}
{\(2x + 2h\)}
{\(x^2 + h\)}
{\(2x + h + 4\)}
{\(x^2 + h + 4\)}
\end{question}

\begin{question}
For \(0 < x < 1,\) which inequality is true?
\choices
{\(\sqrt x < x^2 < x\)}
{\ans{\(x^2 < x < \sqrt x\)}}
{\(\sqrt x < x < x^2\)}
{\(x^2 < \sqrt x < 1\)}
{\(x < \sqrt x < x^2\)}
\end{question}

\begin{question}
Which set of \(x\) satisfies the following system of equations?
\[
\begin{aligned}
    y &= 2 \sin^2 x \\
    y - 1 &= \sin x
\end{aligned}
\]
\choices
{\(\left\{-1, 2\right\}\)}
{\(\left\{-\frac{1}{2}, 1\right\}\)}
{\(\left\{\frac{3 \pi}{2}\right\}\)}
{\ans{\(\left\{\frac{\pi}{2}, \frac{7 \pi}{6}\right\}\)}}
{No solution exists.}
\end{question}

\begin{question}
If \(f(x) = \cos(2x + 4)\) and \(g(x) = \sqrt{x - 2},\) 
then the domain of \(f(g(-2x))\) is 
\choices
{\(x \leqslant -2\)}
{\ans{\(x \leqslant -1\)}}
{\(x \geqslant 1\)}
{\(x \geqslant 2\)}
{all real numbers}
\end{question}

\begin{question}
What values of \(x\) satisfy the inequality \(x^2 - 4x + 3 > 0 \; ?\)
\choices
{\(-\infty < x < \infty\)}
{\(1 < x < 3\)}
{\(1 \leqslant x \leqslant 3\)}
{\ans{\(x < 1, \, x > 3\)}}
{\(x < 1, \, x \geqslant 3\)}
\end{question}

\begin{question}
Which set of \(x\) satisfies \(e^{2x} - 5e^x + 6 = 0 \; ?\)
\choices
{\(\left\{-\ln(3), -\ln(2)\right\}\)}
{\(\left\{ -\ln(2), \ln(3)\right\}\)}
{\(\left\{-\ln \left(\frac{3}{2}\right), -\ln \left(\frac{2}{3}\right)\right\}\)}
{\ans{{\(\left\{ \ln(2), \ln(3)\right\}\)}}}
{\(\left\{2, 3\right\}\)}
\end{question}

\begin{question}
Which expression is equivalent to \(\sec(x) - \csc(x) \; ?\)
\choices
{\(\frac{\cos(x) - \sin(x)}{\sin(x) \cos(x)}\)}
{\(\frac{\sin(x) \cos(x)}{\sin(x) + \cos(x)}\)}
{\(\frac{\sin(x) + \cos(x)}{\sin(x) \cos(x)}\)}
{\ans{\(\frac{\sin(x) - \cos(x)}{\sin(x) \cos(x)}\)}}
{\(\frac{\sin(x) - \cos(x)}{\sin(x) + \cos(x)}\)}
\end{question}

\begin{question}
What values of \(x\) satisfy the equation \(\frac{2}{x + 4} - \frac{2}{x - 1} = 3 \; ?\)
\choices
{\(x = \frac{9 \pm \sqrt{9^2 - 4(3)(2)}}{2(3)}\)}
{\ans{\(x = \frac{9 \pm \sqrt{9^2 - 4(3)(2)}}{2(3)}\)}}
{\(x = \frac{9 \pm \sqrt{9^2 - 4(3)(2)}}{3}\)}
{\(x = \frac{-9 \pm \sqrt{9^2 - 4(3)(2)}}{3}\)}
{\(x = \frac{9 \pm \sqrt{9^2 - (3)(2)}}{3}\)}
\end{question}

\begin{question}
Let \(f\) be an invertible function. The graph of \(y = g(x)\) is obtained by performing the following transformations to the graph of \(y = f(x)\):
\begin{itemize}
    \item 
    The graph is reflected across the line \(y = x.\)
    \item
    The graph is then shifted 4 units to the left.
    \item
    The graph is then translated 3 units down.
\end{itemize}
Which function is \(g(x) \; ?\)
\choices
{\(g(x) = f(x + 4) - 3\)}
{\(g(x) = f(x - 4) - 3\)}
{\ans{\(g(x) = f^{-1}(x + 4) - 3\)}}
{\(g(x) = f^{-1}(x - 4) - 3\)}
{\(g(x) = f^{-1}(x - 4) + 3\)}
\end{question}

\begin{question}
Which expression is equivalent to \(\log(2w) + 3 \log(x) - \frac{1}{2} \log(4y) + 3 \log(z) \; ?\)
\choices
{\(\log \left(\frac{\sqrt y}{wx^3 z^3}\right)\)}
{\(\log \left(\frac{wx^3 z^3}{2 \sqrt y}\right)\)}
{\(\log \left(\frac{w \sqrt y}{4x^3 z^3}\right)\)}
{\(\log \left(\frac{wz^3}{2x^3 \sqrt y}\right)\)}
{\ans{\(\log \left(\frac{wx^3 z^3}{\sqrt y}\right)\)}}
\end{question}